% Format teze zasnovan je na paketu memoir. Prilikom
% zadavanja klase memoir, navedenim opcijama se podešava
% veličina slova (12pt) i jednostrano štampanje (oneside).
\documentclass[12pt,oneside]{memoir}

% Paket koji definiše sve specifičnosti mastera Matematičkog fakulteta.
\usepackage{matfmaster}

% Paket koji obezbeđuje ispravni prikaz ćiriličkih italik slova.
\usepackage{cmsrb}

% Ostali paketi koji se koriste u dokumentu.
\usepackage{listings} % listing programskog koda

% Datoteka sa literaturom u BibTex tj. BibLaTeX/Biber formatu
\bib{hmm_vasovic}

% Ime kandidata na srpskom jeziku (u odabranom pismu)
\autor{Лазар М. Васовић}
% Naslov teze na srpskom jeziku (u odabranom pismu)
\naslov{Скривени Марковљеви модели (\textit{HMM}) у биоинформатици}
% Godina u kojoj je teza predana komisiji
\godina{2021}
% Ime i afilijacija mentora (u odabranom pismu)
\mentor{др Јована \textsc{Ковачевић}, доцент\\ Универзитет у Београду, Математички факултет}
% Ime i afilijacija prvog člana komisije (u odabranom pismu)
\komisijaA{... ... \textsc{...}, ...\\ ..., ...}
% Ime i afilijacija drugog člana komisije (u odabranom pismu)
\komisijaB{... ... \textsc{...}, ...\\ ..., ...}
% Datum odbrane (obrisati ili iskomentarisati ako nije poznat)
\datumodbrane{септембар 2021.}

% Apstrakt na srpskom jeziku (u odabranom pismu)
\apstr{%
...
}

% Ključne reči na srpskom jeziku (u odabranom pismu)
\kljucnereci{биоинформатика, скривени Марковљеви модели (\textit{HMM})}

\begin{document}
% ====================================================================
% Uvodni deo teze
\frontmatter
% ====================================================================
% Naslovna strana
\naslovna
% Strana sa podacima o mentoru i članovima komisije
\komisija
% Strana sa podacima o disertaciji na srpskom jeziku
\apstrakt
% Sadržaj teze
\tableofcontents*

% ====================================================================
% Glavni deo teze
\mainmatter
% ====================================================================

% ------------------------------------------------------------------------------
\chapter{Увод}
% ------------------------------------------------------------------------------
Биоинформатика је интердисциплинарна област која се бави применом рачунарских технологија у области биологије и сродних наука, са нагласком на разумевању биолошких података. Кључна особина јој је управо поменута мултидисциплинарност, која се може представити дијаграмом са слике \ref{fig:venn}.

\begin{figure}[!ht]
  \centering
  \includegraphics[width=0.75\textwidth]{bioinformatika.png}
  \caption{Венов дијаграм интердисциплинарности\cite{venn}}
  \label{fig:venn}
\end{figure}

Овако представљена, биоинформатика је заправо спој статистике, рачунарства и биологије -- сва три истовремено -- по чему надилази појединачне спојеве: биостатистику, науку о подацима и рачунарску биологију. Конкретно, статистички (математички) апаратат служи за рад са подацима, рачунарске технологије тај апарат чине употребљивијим, док биологија даје потребно доменско знање (разумевање) за рад са биолошким и сродним подацима. Иако се може рећи да је биоинформатика, у савременом смислу представљеном приказаним дијаграмом, релативно млада наука, брзо је постала популарна и многи су јој посветили пажњу или се њоме баве \cite{fauziyyah2019, cmero2015, ufpr}.

Међу познатим личностима из овога домена издвајају се научници Филип Компо (\textit{Phillip Compeau}) и Павел Певзнер (\textit{Pavel Pevzner}), аутори књиге \textit{Bioinformatics Algorithms: An Active Learning Approach}. Прво издање књиге изашло је 2014. године, а друго већ наредне, у два тома. Актуелно, треће издање, издато је 2018. године, у једном тому. Захваљујући динамичном и активном приступу биолошким проблемима и њиховим информатичким решењима, као и многим додатним материјалима за учење, књига се користи као уџбеник на више од сто светских факултета\cite{ba}. Међу њима је и Математички факултет Универзитета у Београду, односно на њему доступни мастер курс Увод у биоинформатику, а делови књиге користе се и у настави повезаног мастер и докторског курса Истраживање података у биоинформатици\cite{matf}.

Актуелна иницијатива на нивоу курса Увод у биоинформатику јесте израда електронског уџбеника, заснованог на поменутој књизи. Идеја је да заинтересовани студенти као мастер рад обраде по једно поглавље књиге, при чему обрада укључује писање текста на српском језику, али и имплементацију и евентуалну визуелизацију свих или макар већине пратећих алгоритама. Овај рад настао је управо у склопу представљене иницијативе, међу првима.

Уџбеник кроз једанаест глава обрађује разне теме које су занимљиве у оквиру биоинформатике: почетак репликације (алгоритамско загревање), генске мотиве (рандомизовани алгоритми), асемблирање генома (графовски алгоритми), секвенцирање антибиотика/пептида (алгоритми грубе силе), поређење и поравнање геномских секвенци (динамичко програмирање), блокове синтеније (комбинаторни алгоритми), филогенију (еволутивна стабла), груписање гена (кластеровање), проналажење шаблона (префиксна и суфиксна стабла), откривање гена и мутација секвенце (скривени Марковљеви модели), напредно секвенцирање пептида (рачунарска протеомика). Циљ овог рада је обрада десетог поглавља, заснованог на скривеним Марковљевим моделима\cite{compeau2015}.

Скривени Марковљев модел (у наставку углавном скраћено \textit{HMM}, према енгл. \textit{Hidden Markov Model}), укратко, представља статистички модел који се састоји из следећих елемената: скривених стања ($x_i$), опсервација ($y_i$), вероватноћа прелаза ($a_{ij}$), полазних ($\pi_i$) и излазних вероватноћа ($b_{ij}$), по примеру са слике \ref{fig:hmm}. \textit{HMM} се тако може схватити као коначни аутомат, при чему стања задржавају уобичајено значење, док вероватноће прелаза описују колико се често неки прелаз реализује. Полазне вероватноће одређују почетно стање. Овакав аутомат допуњује се идејом да свако стање са одређеном излазном вероватноћом емитује (приказује) неку опсервацију. Штавише, најчешће су само опсервације и познате у раду са \textit{HMM}, док се позадински низ стања погађа ("предвиђа"), па се управо зато стања и модели називају скривеним\cite{stamp2021}.

\begin{figure}[!ht]
  \centering
  \includegraphics[width=0.75\textwidth]{hmm.png}
  \caption{Једноставан пример скривеног Марковљевог модела\cite{hmm}}
  \label{fig:hmm}
\end{figure}

У претходном пасусу су, наравно, скривени Марковљеви модели представљени малтене само концептуално, на високом нивоу. У наставку ће, међутим, они бити постепено уведени, заједно са мотивацијом за њихову употребу у виду биолошких проблема који се њима решавају. Према идеји електронског уџбеника, излагање ће пратити књигу \textit{Bioinformatics Algorithms: An Active Learning Approach}, а биће имплементирани и сви пратећи алгоритми.

% ------------------------------------------------------------------------------
\chapter{Мотивација}
% ------------------------------------------------------------------------------
За почетак, изложена је мотивација за употребу скривених Марковљевих модела у биоинформатици. Конкретно, представљена су два важна биолошка проблема која се њима могу решити и пратећи појмови из домена, као и једна историјски мотивисана вероватносна мозгалица. Ова глава, дакле, покрива прву петину обрађеног поглавља \textit{Chapter 10: Why Have Biologists Still Not Developed an HIV Vaccine? -- Hidden Markov Models}, и то тачно следеће поднаслове: \textit{Classifying the HIV Phenotype}, \textit{Gambling with Yakuza}, \textit{Two Coins up the Dealer’s Sleeve}, \textit{Finding CG-Islands}, и највећи део додатка из \textit{Detours}.

% Classifying the HIV Phenotype
\section{Погађање фенотипа}
\textit{HIV} је вирус хумане имунодефицијенције, један од најпознатијих вируса, који заражава људе широм света. Својим дугорочним деловањем доводи до смртоносног синдрома стечене имунодефицијенције, познатијег као сида или ејдс. Мада поједини аутори распрострањеност \textit{HIV}-а називају пандемијом, Светска здравствена организација означава је као "глобалну епидемију"\cite{who}.

Постојање \textit{HIV}-а званично је потврђено почетком осамдесетих година двадесетог века, мада се претпоставља да је са примата на људе прешао знатно раније. Недуго по овом открићу, тачније 1984, из америчког Министарства здравља и услуга становништву најављено је да ће вакцина бити доступна кроз наредне две године. Иако до тога није дошло, председник Бил Клинтон је 1997. потврдио да "није питање \textit{да ли} можемо да произведемо вакцину против сиде, већ је просто питање \textit{када} ће до тога доћи". Вакцина, међутим, ни данас није доступна, а многи покушаји су отказани након што се испоставило да кандидати чак повећавају ризик од инфекције код појединих испитаника.

Антивирусне вакцине најчешће се праве од површинских протеина вируса на који се циља, у нади да ће имунски систем, након вакцине, у контакту са живим вирусом знатно брже препознати протеине омотача вируса као стране и уништити их пре него што се вирус намножи у телу. \textit{HIV} је, међутим, карактеристичан по томе што врло брзо мутира, па су његови протеини изузетно варијабилни и није могуће научити имунски систем да исправно одреагује на све мутације. Штавише, може се десити да имунитет научи да исправно реагује само на једну варијанту вируса, а да реакција нема никаквог ефекта на остале варијанте. Овакав имунитет је лошији од имунитета који ништа не зна о вирусу, пошто не покушава да научи ништа ново, што је разлог већ поменуте ситуације да су код неких испитаника вакцине кандитати повећали ризик од заразе. Да ствар буде гора, \textit{HIV} брзо мутира и унутар једне особе, тако да је разлика у узорцима узетих од различитих пацијената увек значајна.

Када се све узме у обзир, као обећавајућа замисао за дизајн свеобухватне вакцине намеће се следећа идеја: идентификовати неки пептид који садржи најмање варијабилне делове површинских протеина свих познатих сојева \textit{HIV}-а и искористити га као основу вакцине. Ни то, међутим, није решење, пошто \textit{HIV} има још једну згодну способност: уме да се сакрије процесом гликолизације. Наиме, протеини омотача су махон гликопротеини, што значи да се након превођења за њих могу закачити многобројни гликански (шећерни) ланци. Овим процесом долази до стварања густог гликанског штита, који омета имунски систем у препознавању вируса. Све досад изнето утиче на немогућност прављења прикладне вакцине у скоријем времену.

Чак и ван контекста вакцине, мутације \textit{HIV}-а прилично су занимљиве за разматрање. Конкретно, најилустративније је бавити се \textit{env} геном, чија је стопа мутације 1--2\% по нуклеотиду годишње. Овај ген кодира два релативно кратка гликопротеина који заједно граде шиљак (спајк) омотача, део вируса задужен за улазак у људске ћелије. Мање важан део шиљка је гликопротеин \textit{gp41} ($\sim$ 345 аминокиселина), док је важнији гликопротеин \textit{gp120} ($\sim$ 480 аминокиселина). О варијабилности другог говори чињеница да на нивоу једног пацијента, у кратком року, скоро половина аминокиселина буде измењено позадинским мутацијама одговарајућег гена, као да је сасвим други протеин.

Ствари постају још занимљивије када се, поред генотипа вируса, разматра и његов фенотип. Примера ради, сваки вирус \textit{HIV}-а може се означити као изолат који ствара синцицијум или као изолат који га не ствара. Након уласка у људску ћелију, гликопротеини омотача могу да изазову спајање заражене ћелије са суседним ћелијама. Резултат тога је синцицијум -- нефункционална вишеједарна ћелијска (цитоплазматична) маса са заједничком ћелијском мембраном. Овакав изолат \textit{HIV}-а означава се као онај који ствара синцицијум и он се тим процесом знатно брже умножава, што даље значи да је опаснији и агресивнији, јер уласком у само једну ћелију убија многе друге.

Испоставља се да је примарна структура гликопротеина \textit{gp120} важан суштински генотипски предиктор фенотипа \textit{HIV}-а. Наиме, узимајући у обзир само низ аминокиселина које чине \textit{gp120}, може се направити једноставан класификатор који погађа да ли проучавани изолат ствара синцицијум или не. Конкретно, научник Жан Жак де Јонг је 1992. анализирао вишеструко поравнање такозване \textit{V3} петље, издвојеног региона у оквиру \textit{gp120}, и формулисао правило 11/25. Према том правилу, сој \textit{HIV}-а највероватније ствара синцицијум уколико му се на 11. или 25. позицији у \textit{V3} петљи налазе аминокиселине аргинин (\textit{R}) или лизин (\textit{K}). Пример мотива \textit{V3} петље дат је на слици \ref{fig:motif}. Приметно је да су управо 11. и 25. позиција међу најваријабилнијим, те да удео критичних \textit{R} и \textit{K} на њима није претерано велик. Наравно, на фенотип утичу и многе друге позиције унутар \textit{gp120} и других протеина.

\begin{figure}[!ht]
  \centering
  \includegraphics[width=0.75\textwidth]{motif.png}
  \caption{Мотив \textit{V3} петље из \cite{compeau2015} генерисан помоћу \cite{weblogo}}
  \label{fig:motif}
\end{figure}

За крај и поенту уводне приче о \textit{HIV}-у, остаје неразрешен још један веома значајан проблем. Како би се уопште разматрало предвиђање фенотипа на основу примарне структуре \textit{gp120}, неопходно је прво доћи до прецизног вишеструког поравнања различитих секвенци аминокиселина. Прво, поравнање мора бити хируршки прецизно, јер нпр. само једна грешка доводи до погрешног податка која вредност је на 11. и 25. позицији \textit{V3} петље. Следеће, неопходно је адекватно обрадити инсерције и делеције, што су врло честе мутације \textit{HIV}-а у многим регионима генома. На крају, потребно је на прави начин оценити квалитет поравнања, нпр. коришћењем различитих матрица скора за сваку појединачну позицију. Ово је донекле могуће урадити коришћењем техника представљених у петом поглављу (\textit{Chapter 5: How Do We Compare DNA Sequences? -- Dynamic Programming}), али уз два главна проблема: алгортми динамичког програмирања су високе сложености и са мање слободе код скорова, а притом не пресликавају најбоље суштину биолошког проблема класификације фенотипа у алгоритамски проблем (фале кораци након поравнања). Постоји потреба за новом формулацијом која би обухватила све што је потребно за статистички потковано поравнање секвенци.

% Finding CG-Islands
\section{Потрага за генима}
...

% Gambling with Yakuza
% Two Coins up the Dealer’s Sleeve
\section{Коцкање са јакузама}
...

% ------------------------------------------------------------------------------
\chapter{Моделовање помоћу \textit{HMM}}
% ------------------------------------------------------------------------------
% Hidden Markov Models
% The Decoding Problem
% Finding the Most Likely Outcome of an HMM
...

% ------------------------------------------------------------------------------
\chapter{Биолошки значај \textit{HMM}}
% ------------------------------------------------------------------------------
% Profile HMMs for Sequence Alignment
% Classifying proteins with profile HMMs
...

% ------------------------------------------------------------------------------
\chapter{Учење \textit{HMM}}
% ------------------------------------------------------------------------------
% Learning the Parameters of an HMM
% Soft Decisions in Parameter Estimation
% Baum-Welch Learning
...

% ------------------------------------------------------------------------------
\chapter{Закључак}
% ------------------------------------------------------------------------------
% The Many Faces of HMMs
% Epilogue: Nature is a Tinkerer and not an Inventor
...

% ------------------------------------------------------------------------------
% Literatura
% ------------------------------------------------------------------------------
\literatura

\end{document}